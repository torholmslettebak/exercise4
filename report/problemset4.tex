% Først spesifiserer vi hvilken dokumentklasse vi vil ha og noen 
% globale opsjoner. Bytt ut 'article' med 'book' hvis du vil ha 
% med kapitler.
\documentclass[a4paper, twoside, titlepage, 11pt]{article}

% Så sier vi fra om hvilke tilleggspakker vi trenger
% til dokumentet vårt. De som du ikke trenger (se kommentaren) 
% kan det være en fordel å kommentere ut (sett prosenttegn foran),
% da vil kompilering gå raskere.

\usepackage[norsk]{babel}             % norske navn rundt omkring
\usepackage[T1]{fontenc}              % norsk tegnsett (æøå)
\usepackage[latin1]{inputenc}         % norsk tegnsett
\usepackage{geometry}                 % anbefalt pakke for å styre marger.
\usepackage{datetime}
\usepackage{amsmath,amsfonts,amssymb} % matematikksymboler
\usepackage{amsthm}                   % for å lage teoremer og lignende.
\usepackage{graphicx}                 % inkludering av grafikk
\usepackage{subfig}                   % hvis du vil kunne ha flere
\usepackage{hyperref}
\usepackage{xcolor}
\usepackage{float}
% \input{preamble/preamble.tex}
% \input{preamble/preamble-addon-listings.tex}
 

                                      % figurer inni en figur
\usepackage{listingsutf8}                 % Fin for inkludering av kildekode

%\usepackage{hyperref}                % Lager hyperlinker i evt. pdf-dokument
                                      % men har noen bugs, så den er kommentert
                                      % bort her.

\usepackage{color}

\definecolor{mygreen}{rgb}{0,0.6,0}
\definecolor{mygray}{rgb}{0.5,0.5,0.5}
\definecolor{mymauve}{rgb}{0.58,0,0.82}
\lstdefinestyle{customc}{
  belowcaptionskip=1\baselineskip,
  breaklines=true,
  frame=L,
  xleftmargin=\parindent,
  language=C,
  showstringspaces=false,
  basicstyle=\footnotesize\ttfamily,
  keywordstyle=\bfseries\color{green!40!black},
  commentstyle=\itshape\color{purple!40!black},
  identifierstyle=\color{blue},
  stringstyle=\color{orange},
}    
\lstdefinestyle{customasm}{
  belowcaptionskip=1\baselineskip,
  frame=L,
  xleftmargin=\parindent,
  language=[x86masm]Assembler,
  basicstyle=\footnotesize\ttfamily,
  commentstyle=\itshape\color{purple!40!black},
}
\lstset{escapechar=@,style=customc}                           
% Indeksgenerering er kommentert ut her. Ta bort prosenttegnene
% hvis du vil ha en indeks:
%\usepackage{makeidx}     
%\makeindex              

% Selve dokumentet begynner:

\begin{document}

% På forsida skal vi ikke ha noen sidenummerering:

% \pagestyle{empty}
% \pagenumbering{roman}

% Inkluder forsida:
% Enkel forside som bruker latex sin \titlepage kommando:
% NB: Bruken av \and mellom navn!
\titlepage
\title{TMA4280 - Problem set 4}
\author{Tor Holm Slettebak}
\date{\today}
\maketitle

% Local Variables:
% TeX-master: "master"
% End:


% Romerske tall på alt før selve rapporten starter er pent.
\pagenumbering{roman}

% For å ikke begynne innholdslista på baksida av forsida:
%\cleardoublepage
% (kun aktuelt når man har twoside som global opsjon)

% Nå vi vil ha noe i topp- og bunnteksten
\pagestyle{headings}

% Si til LaTeX at vi vil ha ei innholdsliste generert akkurat her:
%\tableofcontents

% Pass på at neste side ikke begynner på baksida av en annen side.
%\cleardoublepage

% Arabisk (vanlige tall) sidenummerering. Starter på side 1 igjen.
\pagenumbering{arabic}

% Inkluder alle de andre kildefilene:

% NB: Vi trenger ikke ta med filendelsen .tex her. Den vet
%     LaTeX om selv!

\input{textfile}

% % Dette eksempelet er laget for article-dokumentklassen. Hvis 
% skriver i 'book'-dokumentklassen vil du kanskje bytte ut 
% \section med \chapter, \subsection med \section, osv...

\section{Bakgrunnsteori}

bla.bla bla

\subsection{Integralets opprinnelse}

bla bla bla

\subsection{Riemann-integralet}

bla bla bla


% Local Variables:
% TeX-master: "master"
% End:


% \input{resultater}

% Bibliografi/referanseliste skal komme før appendiks
%\bibliography{kurs}
\bibliographystyle{plain}

% En latex-kommando for å si fra at kapitlene/seksjonene fra nå 
% av skal nummereres med store bokstaver:
\appendix

% \input{appendiks}

% Indeks for rapporten. Ta bort prosenttegn hvis du vil ha det med.
%\printindex

% Avslutter dokumentet vårt:
\end{document}

% Local Variables:
% TeX-master: "master"
% End: